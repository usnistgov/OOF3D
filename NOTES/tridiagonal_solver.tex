\documentclass[10pt,a4paper]{article}

\bibliographystyle{plain}
\usepackage{latexsym}
\usepackage{algorithm,algorithmic}
\usepackage{amssymb,amsthm}
\usepackage{amsmath,amsfonts}
\usepackage{mathrsfs}
\usepackage{enumerate}
\usepackage{graphicx,psfrag}
\usepackage{comment}
%\usepackage[notcite,notref]{showkeys}

\newtheorem{theorem}{Theorem}[section]
\newtheorem{remark}{Remark}[section]
\newtheorem{lemma}{Lemma}[section]
\newtheorem{corollary}{Corollary}[section]
\newtheorem{assumption}{Assumption}[section]
\newtheorem{definition}{{Definition}}[section]
\newtheorem{proposition}{Proposition}[section]
\newtheorem{question}{Question}


\author{G\"unay Do\u gan}
\title{Solving tridiagonal linear systems}

\begin{document}

\maketitle

The Thomas algorithm solves the following tridiagonal system in $O(n)$ time, 
where $n$ is the number of unknowns $\{x_i\}_{i=0}^{n-1}$.
%
\begin{equation}\label{E:tridiag-system}
{\scriptscriptstyle
  \left(\begin{array}{ccccc}
     b_0 & c_0 &        &         &         \\
     a_1 & b_1 &  c_1   &         &         \\
         & a_2 &  b_2   & \ddots  &         \\
         &     & \ddots & \ddots  & c_{n-2} \\
         &     &        & a_{n-1} & b_{n-1}
  \end{array}\right)
  \left(\begin{array}{c}
     x_0 \\ x_1 \\ \vdots \\ \vdots \\ x_{n-1}
  \end{array}\right)
  =
  \left(\begin{array}{c}
     d_0 \\ d_1 \\ \vdots \\ \vdots \\ d_{n-1}
  \end{array}\right).
}
\end{equation}
%
Equation~\eqref{E:tridiag-system} can solved using a simple
procedure that consists of three for-loops executed in linear
time. The three stages of the procedure are: LU-decomposition,
forward substitution and backward substitution. The details
are given in Algorithm~\ref{A:tridiag-solver}. \\
\emph{Note that LAPACK includes an implementation of
a tridiagonal solver, which should be used in practice}.

%
\begin{algorithm}
\caption{The Thomas algorithm}
\label{A:tridiag-solver}
\begin{algorithmic}
\STATE $m_0 = b_0$
\FOR {$i=0,1,\ldots,n-2$}
	\STATE $l_i = a_i / m_i$
	\STATE $m_{i+1} = b_{i+1} - l_i c_i$
\ENDFOR
\STATE $y_0 = d_0$
\FOR {$i=1,2,\ldots,n-1$}
	\STATE $y_i = d_i - l_{i-1} y_{i-1}$
\ENDFOR
\STATE $x_{n-1} = y_{n-1} / m_{n-1}$
\FOR {$i=n-2,n-3,\ldots,0$}
	\STATE $x_i = (y_i - c_i x_{i+1}) / m_i$
\ENDFOR
\end{algorithmic}
\end{algorithm}
%

Unfortunately the Thomas algorithm is not directly applicable 
to coefficient matrices of the form
%
\begin{equation}\label{E:def-periodic-tridiag-A}
A = 
{\scriptscriptstyle
  \left(\begin{array}{ccccc}
     b_0   & c_0 &        &         & a_0     \\
     a_1   & b_1 &  c_1   &         &         \\
           & a_2 &  b_2   & \ddots  &         \\
           &     & \ddots & \ddots  & c_{n-2} \\
    c_{n-1}&     &        & a_{n-1} & b_{n-1}
  \end{array}\right).
}
\end{equation}
%
Nonetheless the Thomas algorithm can still be used as 
a building block for an algorithm to invert $A$.
For this, we define
%
\[
\tilde{A} = 
{\scriptscriptstyle
  \left(\begin{array}{ccccc}
  b_0-a_0& c_0 &        &         &   0     \\
     a_1 & b_1 &  c_1   &         &         \\
         & a_2 &  b_2   & \ddots  &         \\
         &     & \ddots & \ddots  & c_{n-2} \\
      0  &     &        & a_{n-1} & b_{n-1}-c_{n-1}
  \end{array}\right),
}
\quad
u = 
{\scriptscriptstyle
  \left(\begin{array}{c}
     a_0 \\ 0 \\ \vdots \\ \vdots \\ c_{n-1}
  \end{array}\right),
}
\quad
v = 
{\scriptscriptstyle
  \left(\begin{array}{c}
     1 \\ 0 \\ \vdots \\ 0 \\ 1
  \end{array}\right),
}
\]
%
and note 
\[
A = \Tilde{A} + u v^T.
\]
Then the Sherman-Morrison formula enables us to write $A^{-1}$
using $\tilde{A}^{-1}$, which is easier to compute
\[
A^{-1} = \Tilde{A}^{-1} - \frac{\Tilde{A}^{-1} u v^T \Tilde{A}^{-1}}{1 + v^T \Tilde{A}^{-1} u}.
\]
Taking advantage of this, we can write a simple three-step algorithm 
to solve $A x = d$
%
\begin{itemize}
\item Solve $\Tilde{A} z = u$ for $z$.
\item Solve $\Tilde{A} y = d$ for $y$.
\item Compute $x = y - \frac{v^T y}{1 + v^T z} z$.
\end{itemize}
%

\end{document}
